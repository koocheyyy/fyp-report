In recent years, the restaurant industry has experienced a significant shift towards digitization. The move towards web-based restaurant management applications signifies a critical step in this direction, offering a bridge between the traditional dining experience and modern technological conveniences. This final year project aims to design, develop, and evaluate a sophisticated web-based restaurant management application, leveraging contemporary web technologies to enrich the dining experience for customers and streamline restaurant operations.

The application at the core of this project is built upon a React.js frontend and a Django backend, handled by a SQLite database. The decision to use React.js stems from its efficiency in rendering dynamic user interfaces and facilitating an interactive web experience. Django offers comprehensive support for web development needs, including ORM (Object-Relational Mapping) for database interactions, REST Framework for API development, and built-in mechanisms for user authentication and authorization.

One of the application's standout features is its use of QR codes, enabling customers to access the menu and place orders directly from their smartphones. This not only enhances the dining experience by offering convenience and speed but also aligns with contemporary expectations for contactless services. The application further personalizes the customer experience through algorithms that analyze user preferences and order history, delivering tailored menu recommendations that cater to individual tastes and dietary needs. This implementation also solves the problem of splitting bills by introducing a session system to tables which allows customers to order under their own bill or share the bill with others easily all while sitting at the same table.

The backend architecture of the restaurant management system is designed to be efficient. The Django ORM facilitates interactions with the database, abstracting SQL queries into Python. Django REST Framework (DRF) plays a crucial role in this architecture, providing tools for building a RESTful API that the React frontend consumes. This API layer enables seamless communication between the frontend and backend, ensuring that data flows securely and efficiently across the application.

On the frontend, React.js's component-based architecture allows for the development of a modular and interactive user interface. The use of React components enables the reuse of UI elements across the application, ensuring consistency and reducing development time. State management is handled adeptly through React's context API, facilitating the tracking and updating of application state across components without prop drilling. This is crucial for managing user sessions, order details, and authentication states throughout the application. Additionally, the application employs React Router for client-side routing, enhancing the user experience with smooth transitions between views without the need for page reloads.

The database architecture is designed to cater to the needs of restaurant management. It includes models for the restaurant settings, menu items, orders, and customer profiles, among others. Each model is designed to capture data and relationships, enabling the application to offer features such as menu customization, order tracking, and personalized recommendations.

An integral part of the project is the implementation of a recommender algorithm within the UserProfile model. This algorithm is a cornerstone of the application's ability to offer personalized menu recommendations, to enhance the customer experience. By analyzing user preferences and order history, the algorithm identifies preferences, suggesting menu items that align with the user's taste profile. The decision to use a hybrid filtering approach, combining the strengths of collaborative and content-based filtering, underscores the project's commitment to leveraging advanced technologies for personalization.

For restaurant staff, the project includes features for order tracking, menu management, and sales analysis. These tools are designed to improve operational efficiency, enabling staff to manage orders more effectively, update the menu as needed, and analyze sales data to make informed decisions. The inclusion of a kitchen display system further streamlines kitchen operations, displaying orders in real-time and helping staff prioritize based on urgency.

In conclusion, this final year project aims to further improve current implementations and cover some gaps that have not been done before such as the recommender system and the session system.