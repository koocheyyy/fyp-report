\section{Technical Evaluation}
API testing were carried out to the Django backend to measure the performance of the system. The API testing was done using Postman. The custom API requests were made through Postman and the requests have been designed to mimic real world use.

\subsection{API Testing}
Both the server and the database are needed to be tested with multiple requests per second to simulate real life usage scenario. Specifically the Ordering API was tested in this experiment. The reason to this is to test how the server would handle multipe order requests at the same time. Postman was used to send multiple POST requests to the server because the platform allows tests to be written in JavaScript and run in the Postman environment. Provided with the correct body and headers in the API call, every request had to return a 200 response code. The treshold for any response time is 200ms to maintain a responsive system. The tests are separated between POST requests and GET requests. The POST requests are used to send data and the GET requests are used to retrieve from the database.

\subsection{POST Request Testing}
Some of the most demanding POST requests were used in testing and these chosen POST requests are the ones that are most likely to be used in a real world scenario and will have the most impact to the system efficiency and user experience. The POST API calls that are selected to be used in this test are chosen based on the most common user actions. The POST requests that were tested are the following:

    \begin{itemize}
        \item \textbf{Create order}: \url{localhost:8000/api/create_payment_intent/}
        \item \textbf{Generate Invoice}: \url{localhost:8000/api/create-invoice/}
        \item \textbf{Create table session}: \url{localhost:8000/api/sessions/}
    \end{itemize}

\subsection*{Results}
The results of the POST request testing are shown in the table below. The table shows the response time of each request and the response code that was returned. The response time is measured in milliseconds and the response code is the HTTP status code that was returned by the server. The response time is the time taken by the server to process the request and send a response back to the client. The response code is a three-digit code that indicates the status of the request. A response code of 200 indicates that the request was successful, while a response code of 400 or 500 indicates that there was an error.

\subsection{GET Request Testing}
Some of the most demanding GET requests were used in testing and these chosen GET requests are the ones that are most likely to be used in a real world scenario and will have the most impact to the system efficiency and user experience. For example the Recommendation API calls are the most demanding as the recommender algorithm would need to process each new customer that logs in to the digital menu. The GET requests that were tested are the following:

    \begin{itemize}
        \item \textbf{Get all orders}: \url{localhost:8000/api/orders/}
        \item \textbf{Get all menu items}: \url{localhost:8000/api/places/1}
        \item \textbf{Get Food Recommendations}: \url{localhost:8000/api/food-recommendations/}
        \item \textbf{Get Drink Recommendations}: \url{localhost:8000/api/drink-recommendations/}
        \item \textbf{Get Combo Recommendations}: \url{localhost:8000/api/combo-recommendations/}
        \item \textbf{Get all table sessions}: \url{localhost:8000/api/check_session/}
        \item \textbf{Customer Analysis}: \url{localhost:8000/api/customer-analysis/}
        \item \textbf{Total Sales Today}: \url{localhost:8000/api/total-sales-today}
        \item \textbf{Total Sales This Week}: \url{localhost:8000/api/total-sales-week}
        \item \textbf{Total Sales This Month}: \url{localhost:8000/api/total-sales-month}
    \end{itemize}

\subsection*{Results}
The results of the GET request testing are shown in the table below. The table shows the response time of each request and the response code that was returned. The response time is measured in milliseconds and the response code is the HTTP status code that was returned by the server. The response time is the time taken by the server to process the request and send a response back to the client. The response code is a three-digit code that indicates the status of the request. A response code of 200 indicates that the request was successful, while a response code of 400 or 500 indicates that there was an error.

\section{Requirements Evaluation}

\section{Comparative Evaluation}

\section{User Evaluation}