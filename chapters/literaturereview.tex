\section{Evolution of Restaurant Management Systems}
The evolution of restaurant management systems has seen a shift from traditional paper-based methods to digital systems, driven by the need for improved efficiency and customer experience. Traditional systems, such as paper-based menu cards and manual order taking, have limitations in terms of time consumption, manual errors, and customer dissatisfaction \cite{1}. The adoption of digital systems, such as tablet food ordering and digital based ordering, has aimed to address these limitations by providing cost and time efficiency benefits for both management and customers \cite{1}.
Factors driving the adoption of digital restaurant management systems include the need for faster services, reduced dependency on waiters, which increases productivity \cite{3}. Additionally, the use of technology to replace some of the jobs done by human beings and make machines do the work has been a driving factor in the adoption of online food ordering management systems \cite{4}.

The adoption of digital systems has also been influenced by the increasing trend of consumers adopting a more tech-savvy approach to conducting business transactions and leisure activities, leading to the need for restaurant owners to keep up with technological advancements to attract and retain a broader customer base \cite{5}.
Overall, the evolution of restaurant management systems from traditional to digital has been driven by the need for improved efficiency, reduced manual errors, and enhanced customer satisfaction, while also aligning with the changing technological landscape and consumer preferences.

\section{Digital Ordering Systems}
The implementation of digital ordering systems in restaurants has been shown to have a significant impact on customer experience and business outcomes. The use of QR codes for ordering has been designed to provide more advantages, including electronic payment of bills and entertainment facilities, and to reduce the time between ordering and delivering goods to customers \cite{6}. The system allows customers to place food orders by scanning the QR code on the restaurant table, which then directs them to a digital version of the restaurant's menu, enabling them to place orders directly from their phones \cite{2}. This automation of the ordering process has been found to reduce the time of order registry to delivery, improving customer satisfaction and business efficiency \cite{6}.
The use of digital menus in restaurants has been shown to provide several benefits, including the elimination of traditional ordering stages, more favourable choices for customers, and the ability to pay bills digitally, which prevents the pollution of money exchange and has a significant effect on protecting the environment due to the reduced use of paper \cite{6}. The system also allows restaurant owners to have an insightful view of their business data, such as sales data, which can improve decision-making and forecasting demand using data analysis techniques \cite{2}.

Research has shown that the implementation of digital ordering systems has led to increased customer satisfaction, improved table turnover, and reduced labour and menu costs \cite{5}. Customers have reported that the QR menu ordering systems provide convenience, value, and enjoyment, leading to increased customer attraction and satisfaction \cite{5}. Furthermore, the system has been found to improve table turnover and reduce labour and menu costs, leading to improved business efficiency \cite{5}.
In conclusion, the implementation of QR code ordering systems in restaurants has had a positive impact on customer experience and business outcomes, leading to improved customer satisfaction, business efficiency, and insightful views of business data for restaurant owners.

\section{Customer Behavior and Technology Acceptance Towards Digitalisation}
The Unified Theory of Acceptance and Use of Technology (UTAUT) model has been applied to understand customer willingness to use QR code ordering systems in luxury restaurants in Xi'an, China \cite{5}. The UTAUT model contains four independent variables: performance expectancy, effort expectancy, social influence, and facilitating conditions 1.2. Additionally, the UTAUT 2 model, an extended version of UTAUT, includes hedonic motivation, price value, trust, experience, and habit as independent variables, focusing more on the individual use of technology \cite{5}. The study found that performance expectancy, effort expectancy, social influence, facilitating conditions, hedonic motivation, price value, and trust positively affect customer intention to use QR menu ordering \cite{5}. However, effort expectancy and facilitating conditions negatively affect customer behavioural intention \cite{5}.

Factors influencing customer behaviour and acceptance include performance expectancy (the degree to which technology will benefit consumers performing certain activities), effort expectancy (the degree of ease associated with consumers' use of technology), social influence, facilitating conditions, hedonic motivation, price value, and trust \cite{5}. These factors have been found to significantly influence customers' behavioural intention to use QR menu ordering systems \cite{5}.

Moreover, the study revealed that trust is an important variable affecting customers' behavioural intention to use the QR menu, especially when they are asked to provide confidential information such as transaction codes or personal details \cite{5}. This highlights the significance of trust in influencing customer acceptance of QR code ordering systems.

\section{Challenges and Solutions}
The operational challenges faced by restaurants in implementing and managing web-based systems include difficulties in managing customer orders during peak hours, the need for efficient labour scheduling, and the management of customer reservations and waitlists \cite{7}. 

Potential solutions to these challenges involve the implementation of a self-ordering System, which can automate the ordering process, provide real-time order tracking, and manage customer reservations efficiently \cite{7}.

Common technological barriers in implementing web-based systems include compatibility issues with existing systems, scalability concerns, and the need for secure data exchange between different systems \cite{2}.

These barriers can be overcome by using open-source technologies to maintain low costs, ensuring that the system is scalable to accommodate many users, and implementing secure data exchange protocols to protect customer and business data \cite{2}. 

\section{Current Implementations}
There are some QR code ordering solutions reviewed, including solutions such as Toast \cite{8}, Square \cite{9}, Zuppler \cite{10}, TouchBistro \cite{11}, Bbot \cite{12}, Menufy \cite{13}, and Future Ordering \cite{14}. These solutions provide a a lot of functionalities, such as online ordering, and loyalty programs, to specialized tools for creating and managing QR code-based menus. To summarise it, each solution presents features such as real-time menu updates, and data analytics. These solutions highlights the importance of selecting a system that aligns with a restaurant's specific operational needs, budget, and customer engagement goals, emphasizing the role of digital technology in transforming the traditional dining model into a more interactive, convenient, and streamlined process.

Future Ordering \cite{14} is one of the most notable solutions in the UK as several big restaurant names such as KFC, Nandos and Burger King have implemented them. Future Ordering provides a digital ordering platform for food and beverage businesses, focusing on app, web, and kiosk channels. Their system supports various user journeys, including curbside, click’n’collect, and table ordering, and offers comprehensive management tools for digital channels. However, Future Ordering requires you to have their proprietary hardware which can be a barrier in terms of cost for smaller restaurants.

\section{Research Gaps}

While the adoption of digital ordering systems and the implementation of technologies like QR codes have greatly improved the efficiency of order-taking and the overall customer experience, there is a significant gap regarding personalized customer experiences through these digital systems. Most existing systems, as discussed, focus on streamlining operations, reducing errors, improving table turnover, and providing insights into business data. However, there's a scarcity of discussion around how these systems can leverage customer data to offer personalized dining experiences.

Personalization in restaurant management systems can significantly enhance customer satisfaction and loyalty by making customers feel valued and understood. By tracking previous orders and preferences, restaurants can offer tailored recommendations, which not only improves the customer experience but can save time during ordering as the recommendations are more likely to be accepted.

Another gap is the lack of solution for bill splitting. Bill splitting is a common issue in restaurants, especially when dining in groups. It can be a time-consuming and error-prone process for both customers and staff. A digital ordering system that can automate the bill splitting process can significantly improve the customer experience and reduce the workload on staff.

\section{Conclusion}
To conclude this literature review, the evolution of restaurant management systems has seen a shift from traditional paper-based methods to digital systems, driven by the need for improved efficiency and customer experience. The adoption of digital systems has been influenced by the increasing trend of consumers adopting a more tech-savvy approach to conducting business transactions and leisure activities, leading to the need for restaurant owners to keep up with technological advancements to attract and retain a broader customer base. 

The implementation of digital ordering systems in restaurants has been shown to have a significant impact on customer experience and business outcomes, leading to improved customer satisfaction, business efficiency, and insightful views of business data for restaurant owners. The Unified Theory of Acceptance and Use of Technology (UTAUT) model has been applied to understand customer willingness to use QR code ordering systems in luxury restaurants in Xi'an, China [5]. Factors influencing customer behaviour and acceptance include performance expectancy, effort expectancy, social influence, facilitating conditions, hedonic motivation, price value, and trust. These factors have been found to significantly influence customers' behavioural intention to use QR menu ordering systems.

The operational challenges faced by restaurants in implementing and managing web-based systems include difficulties in managing customer orders during peak hours, the need for efficient labour scheduling, and the management of customer reservations and waitlists. Common technological barriers in implementing web-based systems include compatibility issues with existing systems, scalability concerns, and the need for secure data exchange between different systems.

There are some QR code ordering solutions reviewed, including solutions such as Toast, Square, Zuppler, TouchBistro, Bbot, Menufy, and Future Ordering. These solutions provide a lot of functionalities, such as online ordering, and loyalty programs, to specialized tools for creating and managing QR code-based menus. 

However, there is a significant gap regarding personalized customer experiences through these digital systems and the lack of solution for bill splitting. Personalization in restaurant management systems can significantly enhance customer satisfaction and loyalty by making customers feel valued and understood. By tracking previous orders and preferences, restaurants can offer tailored recommendations, which not only improves the customer experience but can save time during ordering as the recommendations are more likely to be accepted. A digital ordering system that can automate the bill splitting process can significantly improve the customer experience and reduce the workload on staff. This research aims to address these gaps by proposing a digital ordering system that leverages customer data to offer personalized dining experiences and automate the bill splitting process. The next chapter will discuss the research methodology used to achieve this goal.


