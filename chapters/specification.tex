\section{Empirical Exploration}
This section describes the preparatory empirical work performed for this project. It aims to define the scope of the final solution by presenting a preliminary discussion with restaurant owners and staff about their current practices and needs. The goal is to understand the current state of the art in the restaurant industry and to identify the main challenges and opportunities for the development of a new system. This section also presents the results of a preliminary survey conducted with potential users of the system, which aims to confirm the research gaps and the interest of new technology.

\subsection*{Interviews with Restaurant Owners and Staff}
The first step in the empirical exploration was to conduct interviews with restaurant owners and staff. The goal was to understand the current state of the art in the restaurant industry and to identify the main challenges and opportunities for the development of a new system. The interviews were conducted with the help of a semi-structured questionnaire, which was designed to guide the conversation and ensure that all relevant topics were covered. The questionnaire was divided into three main sections: the current practices and challenges of the restaurant, the potential benefits and opportunities of a new system, and the main requirements and expectations for the system. The interviews were conducted with a total of 2 restaurant owners, one is the owner of a small restaurant in Portugal and the other is the owner of a multi-branch restaurant in Malaysia, and 6 restaurant staffs.

\subsection*{Interview with Restaurant Owners}
The interviews with the restaurant owners aimed to identify their perspective of a restaurant owner to understand the current practices and challenges of the restaurant, the potential benefits and opportunities of a new system, and the main requirements and expectations for the system. Some of the questions that were asked during the interviews include:
\begin{itemize}
    \item What are the main challenges that you face in managing your restaurant?
    \item What are the main benefits that you expect from a new system?
    \item What are the main requirements and expectations that you have for the system?
    \item What are their current practices in managing the restaurant?
    \item What are the other opportunities that you think a new system can bring to your restaurant?
    \item How do they deal with the bill splitting problem?
    \item What do they think about having a system that would be able to help them with bill splitting?
    \item What do they think about having personalised digital menus for their customers?
    \item What do they think about my idea of the system?
\end{itemize}

\subsection*{Interview with Restaurant Staff}
Similar questions were asked to the restaurant staffs, but the questions were tailored to their roles and responsibilities in the restaurant. The interviews aimed to identify the main challenges and opportunities that the staff face in their daily work, and to understand their main requirements and expectations for the system. Some of the questions that were asked during the interviews include:
\begin{itemize}
    \item What are the main challenges that you face in your daily work?
    \item What are the main benefits that you expect from a new system?
    \item What are the main requirements and expectations that you have for the system?
    \item What are their current practices in managing the restaurant?
    \item What are the other opportunities that you think a new system can bring to your restaurant?
    \item How do they deal with group orders and bill splitting?
    \item What do they think about having a system that would be able to help them with bill splitting?
    \item What do they think about having personalised digital menus for their customers?
    \item What do they think about my idea of the system?
\end{itemize}

\subsection*{Surveys With Customers}
This survey aimed to understand dining habits, preferences, and challenges faced by people when eating out, particularly focusing on the use of QR code ordering systems, group ordering dynamics, bill splitting difficulties, and interest in technological solutions to these issues. The respondents varied in their frequency of dining out, ranging from daily to a few times a month. The survey was conducted with a total of 43 respondents. The survey questions are as follows:

\begin{itemize}
    \item How often do you dine out at restaurants?
    \item Have you ever ordered food in a restaurant using a QR code?
    \item If yes, how would you rate the experience? (Only answer if you selected "Yes" above.)
    \item When dining with others, how do you prefer to place your order?
    \item What challenges have you faced when ordering food in restaurants?
    \item How do you currently split bills when dining out with friends or family?
    \item How challenging is it to split bills when ordering as a group?
    \item How interested would you be in an app that allows you to order individually or as a group and easily split the bill directly through the app?
    \item Would you be interested in a feature that recommends menu items based on your past orders?
    \item Do wish that restaurants have a personalized menu just for your preferences?
\end{itemize}

A large portion of respondents has utilized QR code ordering systems, with experiences rating from "Very Satisfactory" to "Unsatisfactory", mostly leaning towards Satisfactory. Most prefer to place orders as a group, either opting for one bill for everyone or paying individually, yet the common practice is to calculate everyone's share after a single person pays the bill. Challenges identified include difficulty in getting the server's attention, limited time to decide on the menu, specifying dietary restrictions, and the inconvenience of splitting bills.

Regarding bill splitting, a large number of respondents found it somewhat to very challenging, indicating a potential area for improvement in dining experiences. On top of that, high interest was expressed in an solution that covers individual or group ordering and simplifies bill splitting.

Furthermore, the idea of personalised menu recommendations based on past orders received strong support, all respondents were interested for personalized menus tailored to their preferences.

To conclude this survey, the results suggest that there is a demand for a system that can help with group ordering and bill splitting, and that can provide personalized digital menus to customers. The survey also aligns with the findings from the literature review that there is a growing trend in the use of digital ordering systems in restaurants, and that there is still room for improvement.

\section{Requirements}
This section presents the requirements for the system, which were identified based on the empirical exploration and the literature review. The requirements are divided into functional and non-functional requirements.

\subsection*{Functional Requirements}
The functional requirements describe the main features and capabilities of the system. They were identified based on the interviews with restaurant owners and staff, and the survey with customers. The main functional requirements for the system are as follows:

\begin{itemize}
    \item \textbf{QR Code Ordering:} The system should allow customers to place orders by scanning a QR code on their table, which will take them to a digital menu.
    \item \textbf{Group Ordering:} Use sessions to allow customers to place orders as a group, and to split the bill easily. Additional customers can join the existing sessions or create a new session.
    \item \textbf{Bill Splitting:} The adminstrator would be able to see all session orders on a table and would be able to split the bill automatically.
    \item \textbf{Personalised Digital Menus:} The system should provide customers with personalised digital menus, which recommend menu items based on their past orders and preferences.
    \item \textbf{Shopping Cart:} The system should allow customers to add items to a shopping cart, and to review and modify their orders before submitting them. The order is submitted to the server containing the table number and the user's session.
    \item \textbf{Table Management:} The adminstrator would be able to add and remove tables, and print QR codes for each table.
    \item \textbf{Order Management:} The adminstrator would be able to view all orders from all the tables and track status of the orders. The system would show the time since the order was placed.
    \item \textbf{Kitchen Display System:} The system should provide a kitchen display system to help kitchen staff manage orders and track their status. Have separate views for different sections of the kitchen. Example, Bar and Kitchen.
    \item \textbf{Floor Staff Management:} The system should provide a floor staff orders that needs to be served and the table numbers of orders.
    \item \textbf{Menu Management:} The system should allow restaurant staff to manage the digital menu, add new items, update prices, and remove items as needed.
    \item \textbf{Customer Management:} The system should allow restaurant staff to manage customer accounts, view customer preferences and past orders, and update customer information as needed.
    \item \textbf{Settings:} The system should allow restaurant staff to configure settings such as tax rates, name of restaurant, restaurant logo, menu theme.
    \item \textbf{Analytics:} The system should provide restaurant owners with analytics and reports on customer orders, preferences, and trends, to help them make data-driven decisions.

\end{itemize}

\subsection*{Non-Functional Requirements}
The non-functional requirements describe the quality attributes of the system, such as performance, security, and usability.

\begin{itemize}
    \item \textbf{Performance:} The system should be able to handle a large number of users and orders, and should be able to process orders quickly and efficiently.
    \item \textbf{Security:} The system should be secure, and should protect customer data and payment information. It should also be able to prevent unauthorized access to the system. There should be a clear separation between the customer and the restaurant staff.
    \item \textbf{Usability:} The system should be easy to use, and should provide a user-friendly interface for both customers and restaurant staff.
    \item \textbf{Reliability:} The system should be reliable, and should be able to handle errors and failures gracefully. It should also be able to recover from failures quickly and without data loss.
    \item \textbf{Scalability:} The system should be able to scale to accommodate a growing number of users and orders, and should be able to handle increased demand without a significant decrease in performance.
    \item \textbf{Accessibility:} The system should be accessible to people with disabilities, and should provide support for assistive technologies such as screen readers and voice recognition software.
    \item \textbf{Compatibility:} The system should be compatible with a wide range of devices and operating systems, and should be able to run on both desktop and mobile platforms.
    \item \textbf{Maintainability:} The system should be easy to maintain, and should be able to accommodate changes and updates without a significant impact on the system's performance or reliability.
    \item \textbf{Cost:} The system should be cost-effective, and should provide good value for money. It should also be able to reduce costs for restaurant owners and staff, and should be able to increase revenue and customer satisfaction.
    \item \textbf{Ease of Deployment:} The system should be easy to deploy, and should be able to run on a wide range of hardware and software platforms. It should also be able to integrate with existing systems and technologies.
    \item \textbf{Environmental:} The system should be environmentally friendly, and should be able to reduce waste and energy consumption. It should also be able to promote sustainable practices and support environmental conservation efforts.
    
\end{itemize}

\section{Prototyping}
This section presents the prototyping process for the system, which was performed to validate the requirements and to explore the main features and capabilities of the system.
