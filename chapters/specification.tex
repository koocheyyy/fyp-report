\section{Empirical Exploration}
This section describes the preparatory empirical work performed for this project. It aims to define the scope of the final solution by presenting a preliminary discussion with restaurant owners and staff about their current practices and needs. The goal is to understand the current state of the art in the restaurant industry and to identify the main challenges and opportunities for the development of a new system. This section also presents the results of a preliminary survey conducted with potential users of the system, which aims to identify the main requirements and expectations of the system.

\subsection{Interviews with Restaurant Owners and Staff}
The first step in the empirical exploration was to conduct interviews with restaurant owners and staff. The goal was to understand the current state of the art in the restaurant industry and to identify the main challenges and opportunities for the development of a new system. The interviews were conducted with the help of a semi-structured questionnaire, which was designed to guide the conversation and ensure that all relevant topics were covered. The questionnaire was divided into three main sections: the current practices and challenges of the restaurant, the potential benefits and opportunities of a new system, and the main requirements and expectations for the system. The interviews were conducted with a total of 2 restaurant owners, one is the owner of a small restaurant and the other is the owner of a multi-branch restaurant, and 6 restaurant staffs.


